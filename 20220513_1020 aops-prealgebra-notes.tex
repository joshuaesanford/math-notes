\documentclass[letterpaper, 12pt]{report}
\usepackage{enumitem}
\usepackage{multicol}
\usepackage{graphicx}
\usepackage{ulem} % allows for normal strikethroughs using \sout{}
\usepackage{cancel}
\usepackage{amsmath}
\usepackage{setspace}
\usepackage{geometry}
\geometry{
letterpaper,
left=25mm,
top=20mm,
bottom=25mm
}


\setstretch{1.25}

\begin{document}

\

\vspace{-0.3cm}

\begin{center}
\LARGE Prealgebra Key Concepts \\
\end{center}

\

\vspace{-0.5cm}

\begin{flushleft}

\textit{Properties of Addition.} \\
\vspace{0.2cm}
\hspace{0.5cm} Commutative Property: ~~~~$a + b = b + a$\\
\hspace{0.5cm} Associative Property: ~~~~~~$(a + b) + c = a + (b + c)$\\
\hspace{0.5cm} Additive Identity: ~~~~~~~~~~~~$a + 0 = a$ \\
\hspace{0.5cm} Additive Inverse: ~~~~~~~~~~~~~$a - a = 0$\\

\

\textit{Properties of Multiplication.} \\
\vspace{0.2cm}
\hspace{0.5cm} Commutative Property: ~~~~~~~~~~~~$ab = ba$\\
\hspace{0.5cm} Associative Property: ~~~~~~~~~~~~~~~$(ab)c = a(bc)$ \\
\hspace{0.5cm} Multiplicative Identity: ~~~~~~~~~~~~ $a \cdot 1 = a$ \\
\hspace{0.5cm} Distributive Property: ~~~~~~~~~~~~~ $a(b + c) = ab + ac$ \\
\hspace{0.5cm} Multiplication Property of Zero: ~ $a \cdot 0 = 0$ \\

\

\textit{Order of Operations.} \\
\vspace{0.2cm}
\textbf{1.} Evaluate expressions inside of parentheses first \\
\textbf{2.} Compute Powers \\
\textbf{3.} Multiply and divide from left to right \\
\textbf{4.} Add and subtract from left to right \\

\

\textit{Factoring.} \\
\vspace{0.2cm}
$ab + ac = a(b + c)$ \\

\

\textit{Negation.} \\
\vspace{0.2cm}
Negation means taking a number and flipping its sign so that if it were added to itself (before the sign flip) the result would be zero. For example, the negation of a number $x$ is $-x$ where $-x + x = 0$. Note that $x$ can be negative. For example, if $x = -2$ then $-x + x = -(-2) + (-2) = 2 + (-2) = 0$. Therefore, the negation of $-2$ is $2$.

\

When adding we imagine movement on a number line. When a \textit{positive} number $y$ is added to a number $x$ you start on $x$ and move to the \textit{right} an amount indicated by the magnitude of $y$. When a \textit{negative} number $y$ is added to a number $x$ you start on $x$ and move to the \textit{left} an amount indicated by the magnitude of $y$. \\
\pagebreak

\textit{Negation of Negation.}\\
\vspace{0.2cm}
$-(-x) = x$\\

\

\textit{Multiplying by -1.}\\
\vspace{0.2cm}
$(-1)x = -x$ \\

\

\textit{Multiplying by Negation.} \\
\vspace{0.2cm}
$(-x)y = -(xy)$ \\
$x(-y) = -(xy)$ \\

\

\textit{Product of Two Negations.} \\
\vspace{0.2cm}
$(-x)(-y) = xy$ \

\

\textit{Negation of Sum.}\\
\vspace{0.2cm}
$-(x + y) = (-x) + (-y)$ \\

\

\textit{Subtraction.}\\
\vspace{0.2cm}
$a - b = a + (-b)$ \\

\

Subtraction by a number means adding its negation. Negation and subtraction look the same but are different operations. Negation takes one input and returns its reflection across zero on the number line. Subtraction takes two inputs and returns the sum of the first input with the negation of the second. \\

\

\textit{Subtracting from zero.} \\
\vspace{0.2cm}
$0 - x = -x$ \\

\

\textit{Self-Subtraction.} \\
\vspace{0.2cm}
$x - x = 0$ \\

\

\textit{Subtracting Zero.} \\
\vspace{0.2cm}
$x - 0 = x$ \\

\pagebreak

\textit{Subtraction of Negation.} \\
\vspace{0.2cm} 
$x - (-y) = x + y$ \\

\

\textit{Subtraction from Negation.} \\
\vspace{0.2cm}
$-x - y = -(x + y)$ \\

\

\textit{Negation of Subtraction.} \\
\vspace{0.2cm}
$-(x - y) = -x + y$ \\

\

\textit{Subtraction is neither commutative nor associative.} \\
\vspace{0.2cm}
$(a - b) - c$ \hspace{0.2cm} \textit{is not necessarily equal to} \hspace{0.2cm} $a - (b - c)$ \\
\vspace{0.2cm}
$a - b$ \hspace{0.2cm} \textit{is not necessarily equal to} \hspace{0.2cm} $b - a$ \\

\

\textit{To solve subtraction problems:}\\
\vspace{0.2cm}
\textbf{1.} Change all subtractions to additions \\
\textbf{2.} Rearrange additions using commutative and associative properties \\
\textbf{3.} [\textit{Optional}] Change some additions back to subtractions. \\

\

\textit{Multiplication distributes over subtraction.} \\
\vspace{0.2cm}
$a(b - c) = ab - ac$ \\

\

\textit{Reciprocal of $x$.}\\
\vspace{0.3cm}
$\dfrac{1}{x}$ is called the \textit{reciprocal} or \textit{multiplicative inverse} of $x$. \\

\vspace{0.3cm}

$\dfrac{1}{x} \cdot x = x^{-1}x = 1$ \\

\vspace{0.3cm}

The reciprocal of zero is undefined. \\

\

\textit{Reciprocal of Reciprocal.} \\
\vspace{0.3cm}
$\dfrac{1}{\frac{1}{x}} = (x^{-1})^{-1} = x$\\

\

\textit{Reciprocal of Product.} \\
\vspace{0.3cm}
$(xy)^{-1} = x^{-1}y^{-1} = \dfrac{1}{xy}$
\pagebreak

\textit{Reciprocal of Negation.} \\
\vspace{0.2cm}
$(-x)^{-1} = -(x)^{-1}$ \\

\

\textit{Division.}\\
\vspace{0.3cm}
$a \div b = a \cdot \dfrac{1}{b}$\\

\

Division is an operation that take two inputs and multiplies the first input by the reciprocal of the second. Division by zero is undefined because the reciprocal of zero is undefined. \\

\

\textit{Dividing into Zero.} \\
\vspace{0.3cm}
$\dfrac{0}{x} = 0$ \\

\

\textit{Self-Division.}\\
\vspace{0.3cm}
$\dfrac{x}{x} = 1$ \\

\

\textit{Dividing by 1.}\\
\vspace{0.3cm}
$\dfrac{x}{1} = x$ \\

\

\textit{Dividing by Reciprocal.}\\
\vspace{0.3cm}
$x \div \dfrac{1}{y} = \dfrac{x}{\frac{1}{y}} = \dfrac{x}{y^{-1}} = xy$ \\

\

\textit{Division into Negation.} \\
\vspace{0.3cm}
$\dfrac{(-x)}{y} = -\dfrac{x}{y}$

\

\textit{Division by Negation.} \\
\vspace{0.3cm}
$\dfrac{x}{-y} = -\dfrac{x}{y}$ \\

\

\textit{Negation Divided by Negation.}\\
\vspace{0.3cm}
$\dfrac{(-x)}{(-y)} = \dfrac{x}{y}$ \\

\pagebreak

\textit{Division is neither commutative nor associative.} \\
\vspace{0.3cm}
$\dfrac{a}{b} \neq \dfrac{b}{a}$ \hspace{0.2cm} when \hspace{0.2cm} $a \neq b$ \\

\

$(a \div b) \div c$ \hspace{0.2cm} \textit{is not necessarily equal to} \hspace{0.2cm} $a \div (b \div c)$

\end{flushleft}

\end{document}









